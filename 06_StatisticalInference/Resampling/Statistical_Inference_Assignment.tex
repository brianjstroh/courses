\documentclass[]{article}
\usepackage{lmodern}
\usepackage{amssymb,amsmath}
\usepackage{ifxetex,ifluatex}
\usepackage{fixltx2e} % provides \textsubscript
\ifnum 0\ifxetex 1\fi\ifluatex 1\fi=0 % if pdftex
  \usepackage[T1]{fontenc}
  \usepackage[utf8]{inputenc}
\else % if luatex or xelatex
  \ifxetex
    \usepackage{mathspec}
  \else
    \usepackage{fontspec}
  \fi
  \defaultfontfeatures{Ligatures=TeX,Scale=MatchLowercase}
\fi
% use upquote if available, for straight quotes in verbatim environments
\IfFileExists{upquote.sty}{\usepackage{upquote}}{}
% use microtype if available
\IfFileExists{microtype.sty}{%
\usepackage{microtype}
\UseMicrotypeSet[protrusion]{basicmath} % disable protrusion for tt fonts
}{}
\usepackage[margin=1in]{geometry}
\usepackage{hyperref}
\hypersetup{unicode=true,
            pdftitle={Statistical Inference Assignment},
            pdfauthor={Brian Stroh},
            pdfborder={0 0 0},
            breaklinks=true}
\urlstyle{same}  % don't use monospace font for urls
\usepackage{color}
\usepackage{fancyvrb}
\newcommand{\VerbBar}{|}
\newcommand{\VERB}{\Verb[commandchars=\\\{\}]}
\DefineVerbatimEnvironment{Highlighting}{Verbatim}{commandchars=\\\{\}}
% Add ',fontsize=\small' for more characters per line
\usepackage{framed}
\definecolor{shadecolor}{RGB}{248,248,248}
\newenvironment{Shaded}{\begin{snugshade}}{\end{snugshade}}
\newcommand{\KeywordTok}[1]{\textcolor[rgb]{0.13,0.29,0.53}{\textbf{#1}}}
\newcommand{\DataTypeTok}[1]{\textcolor[rgb]{0.13,0.29,0.53}{#1}}
\newcommand{\DecValTok}[1]{\textcolor[rgb]{0.00,0.00,0.81}{#1}}
\newcommand{\BaseNTok}[1]{\textcolor[rgb]{0.00,0.00,0.81}{#1}}
\newcommand{\FloatTok}[1]{\textcolor[rgb]{0.00,0.00,0.81}{#1}}
\newcommand{\ConstantTok}[1]{\textcolor[rgb]{0.00,0.00,0.00}{#1}}
\newcommand{\CharTok}[1]{\textcolor[rgb]{0.31,0.60,0.02}{#1}}
\newcommand{\SpecialCharTok}[1]{\textcolor[rgb]{0.00,0.00,0.00}{#1}}
\newcommand{\StringTok}[1]{\textcolor[rgb]{0.31,0.60,0.02}{#1}}
\newcommand{\VerbatimStringTok}[1]{\textcolor[rgb]{0.31,0.60,0.02}{#1}}
\newcommand{\SpecialStringTok}[1]{\textcolor[rgb]{0.31,0.60,0.02}{#1}}
\newcommand{\ImportTok}[1]{#1}
\newcommand{\CommentTok}[1]{\textcolor[rgb]{0.56,0.35,0.01}{\textit{#1}}}
\newcommand{\DocumentationTok}[1]{\textcolor[rgb]{0.56,0.35,0.01}{\textbf{\textit{#1}}}}
\newcommand{\AnnotationTok}[1]{\textcolor[rgb]{0.56,0.35,0.01}{\textbf{\textit{#1}}}}
\newcommand{\CommentVarTok}[1]{\textcolor[rgb]{0.56,0.35,0.01}{\textbf{\textit{#1}}}}
\newcommand{\OtherTok}[1]{\textcolor[rgb]{0.56,0.35,0.01}{#1}}
\newcommand{\FunctionTok}[1]{\textcolor[rgb]{0.00,0.00,0.00}{#1}}
\newcommand{\VariableTok}[1]{\textcolor[rgb]{0.00,0.00,0.00}{#1}}
\newcommand{\ControlFlowTok}[1]{\textcolor[rgb]{0.13,0.29,0.53}{\textbf{#1}}}
\newcommand{\OperatorTok}[1]{\textcolor[rgb]{0.81,0.36,0.00}{\textbf{#1}}}
\newcommand{\BuiltInTok}[1]{#1}
\newcommand{\ExtensionTok}[1]{#1}
\newcommand{\PreprocessorTok}[1]{\textcolor[rgb]{0.56,0.35,0.01}{\textit{#1}}}
\newcommand{\AttributeTok}[1]{\textcolor[rgb]{0.77,0.63,0.00}{#1}}
\newcommand{\RegionMarkerTok}[1]{#1}
\newcommand{\InformationTok}[1]{\textcolor[rgb]{0.56,0.35,0.01}{\textbf{\textit{#1}}}}
\newcommand{\WarningTok}[1]{\textcolor[rgb]{0.56,0.35,0.01}{\textbf{\textit{#1}}}}
\newcommand{\AlertTok}[1]{\textcolor[rgb]{0.94,0.16,0.16}{#1}}
\newcommand{\ErrorTok}[1]{\textcolor[rgb]{0.64,0.00,0.00}{\textbf{#1}}}
\newcommand{\NormalTok}[1]{#1}
\usepackage{graphicx,grffile}
\makeatletter
\def\maxwidth{\ifdim\Gin@nat@width>\linewidth\linewidth\else\Gin@nat@width\fi}
\def\maxheight{\ifdim\Gin@nat@height>\textheight\textheight\else\Gin@nat@height\fi}
\makeatother
% Scale images if necessary, so that they will not overflow the page
% margins by default, and it is still possible to overwrite the defaults
% using explicit options in \includegraphics[width, height, ...]{}
\setkeys{Gin}{width=\maxwidth,height=\maxheight,keepaspectratio}
\IfFileExists{parskip.sty}{%
\usepackage{parskip}
}{% else
\setlength{\parindent}{0pt}
\setlength{\parskip}{6pt plus 2pt minus 1pt}
}
\setlength{\emergencystretch}{3em}  % prevent overfull lines
\providecommand{\tightlist}{%
  \setlength{\itemsep}{0pt}\setlength{\parskip}{0pt}}
\setcounter{secnumdepth}{0}
% Redefines (sub)paragraphs to behave more like sections
\ifx\paragraph\undefined\else
\let\oldparagraph\paragraph
\renewcommand{\paragraph}[1]{\oldparagraph{#1}\mbox{}}
\fi
\ifx\subparagraph\undefined\else
\let\oldsubparagraph\subparagraph
\renewcommand{\subparagraph}[1]{\oldsubparagraph{#1}\mbox{}}
\fi

%%% Use protect on footnotes to avoid problems with footnotes in titles
\let\rmarkdownfootnote\footnote%
\def\footnote{\protect\rmarkdownfootnote}

%%% Change title format to be more compact
\usepackage{titling}

% Create subtitle command for use in maketitle
\newcommand{\subtitle}[1]{
  \posttitle{
    \begin{center}\large#1\end{center}
    }
}

\setlength{\droptitle}{-2em}

  \title{Statistical Inference Assignment}
    \pretitle{\vspace{\droptitle}\centering\huge}
  \posttitle{\par}
    \author{Brian Stroh}
    \preauthor{\centering\large\emph}
  \postauthor{\par}
      \predate{\centering\large\emph}
  \postdate{\par}
    \date{September 26, 2018}


\begin{document}
\maketitle

\begin{Shaded}
\begin{Highlighting}[]
\KeywordTok{library}\NormalTok{(dplyr)}
\KeywordTok{library}\NormalTok{(reshape2)}
\KeywordTok{library}\NormalTok{(ggplot2)}
\KeywordTok{library}\NormalTok{(gridExtra)}
\KeywordTok{set.seed}\NormalTok{(}\DecValTok{51}\NormalTok{)}
\end{Highlighting}
\end{Shaded}

\subsection{Part 1: Exponential
Simulation}\label{part-1-exponential-simulation}

\subsubsection{Overview}\label{overview}

In the Exponential Simulation section of this report, we will explore
how the
\href{https://en.wikipedia.org/wiki/Central_limit_theorem}{Central Limit
Theorem} can be applied to a non-normal variable. We will create 1000
samples of 40 random instances from an exponential distribution. Then we
will plot a histogram of the mean and variance of each of the 1000
samples and compare these to the equivalent normal curves.

\subsubsection{Simulations}\label{simulations}

\begin{Shaded}
\begin{Highlighting}[]
\NormalTok{mySample<-}\KeywordTok{matrix}\NormalTok{(}\KeywordTok{rexp}\NormalTok{(}\DataTypeTok{n=}\DecValTok{40}\OperatorTok{*}\DecValTok{1000}\NormalTok{, }\DataTypeTok{rate =}\NormalTok{ .}\DecValTok{2}\NormalTok{),}\DecValTok{40}\NormalTok{,}\DecValTok{1000}\NormalTok{)}
\NormalTok{Means<-}\KeywordTok{colSums}\NormalTok{(mySample)}\OperatorTok{/}\DecValTok{40} 
\CommentTok{#Variance calculated using the formula : Var[x] = E[X^2] - (E[x])^2     }
\CommentTok{#source: https://en.wikipedia.org/wiki/Variance}
\NormalTok{Variances<-}\KeywordTok{colSums}\NormalTok{(mySample}\OperatorTok{^}\DecValTok{2}\NormalTok{)}\OperatorTok{/}\DecValTok{40}\OperatorTok{-}\NormalTok{(}\KeywordTok{colSums}\NormalTok{(mySample)}\OperatorTok{/}\DecValTok{40}\NormalTok{)}\OperatorTok{^}\DecValTok{2}
\NormalTok{mySampleStats<-}\KeywordTok{tbl_df}\NormalTok{(}\KeywordTok{cbind}\NormalTok{(Means,Variances))}
\end{Highlighting}
\end{Shaded}

\subsubsection{Sample Mean versus Theoretical
Mean}\label{sample-mean-versus-theoretical-mean}

\begin{Shaded}
\begin{Highlighting}[]
\KeywordTok{ggplot}\NormalTok{(mySampleStats, }\KeywordTok{aes}\NormalTok{(}\DataTypeTok{x =}\NormalTok{ Means, }\DataTypeTok{mean =} \KeywordTok{mean}\NormalTok{(mySampleStats}\OperatorTok{$}\NormalTok{Means), }\DataTypeTok{sd =} \KeywordTok{sqrt}\NormalTok{(}\KeywordTok{var}\NormalTok{(mySampleStats}\OperatorTok{$}\NormalTok{Means)), }
                          \DataTypeTok{binwidth =}\NormalTok{ .}\DecValTok{2}\NormalTok{, }\DataTypeTok{n =} \KeywordTok{nrow}\NormalTok{(mySampleStats))) }\OperatorTok{+}
\StringTok{      }\KeywordTok{theme_dark}\NormalTok{() }\OperatorTok{+}
\StringTok{      }\KeywordTok{geom_histogram}\NormalTok{(}\DataTypeTok{binwidth =}\NormalTok{ .}\DecValTok{2}\NormalTok{, }\DataTypeTok{colour =} \StringTok{"white"}\NormalTok{, }\DataTypeTok{fill =} \StringTok{"blue"}\NormalTok{, }\DataTypeTok{size =} \FloatTok{0.1}\NormalTok{) }\OperatorTok{+}
\StringTok{      }\KeywordTok{stat_function}\NormalTok{(}\DataTypeTok{fun =} \ControlFlowTok{function}\NormalTok{(x) }\KeywordTok{dnorm}\NormalTok{(x, }\DataTypeTok{mean =} \KeywordTok{mean}\NormalTok{(mySampleStats}\OperatorTok{$}\NormalTok{Means), }
                                            \DataTypeTok{sd =} \KeywordTok{sqrt}\NormalTok{(}\KeywordTok{var}\NormalTok{(mySampleStats}\OperatorTok{$}\NormalTok{Means))) }\OperatorTok{*}\StringTok{ }\KeywordTok{nrow}\NormalTok{(mySampleStats) }\OperatorTok{*}\StringTok{ }\NormalTok{.}\DecValTok{2}\NormalTok{,}
                    \KeywordTok{aes}\NormalTok{(}\DataTypeTok{color =} \StringTok{"Normal_Distribution"}\NormalTok{), }\DataTypeTok{size =}\DecValTok{1}\NormalTok{) }\OperatorTok{+}
\StringTok{      }\KeywordTok{geom_vline}\NormalTok{(}\KeywordTok{aes}\NormalTok{(}\DataTypeTok{xintercept =}\NormalTok{ (}\DecValTok{1}\OperatorTok{/}\NormalTok{.}\DecValTok{2}\NormalTok{), }\DataTypeTok{color =} \StringTok{"Theoretical_Mean"}\NormalTok{), }\DataTypeTok{linetype=}\StringTok{"dashed"}\NormalTok{, }\DataTypeTok{size=}\FloatTok{1.5}\NormalTok{) }\OperatorTok{+}
\StringTok{      }\KeywordTok{geom_vline}\NormalTok{(}\KeywordTok{aes}\NormalTok{(}\DataTypeTok{xintercept =} \KeywordTok{mean}\NormalTok{(mySampleStats}\OperatorTok{$}\NormalTok{Means), }\DataTypeTok{color =} \StringTok{"Sample_Mean"}\NormalTok{), }\DataTypeTok{size=}\DecValTok{1}\NormalTok{) }\OperatorTok{+}
\StringTok{      }\KeywordTok{scale_color_manual}\NormalTok{(}\DataTypeTok{name =} \StringTok{"Chart Element"}\NormalTok{, }\DataTypeTok{values =} \KeywordTok{c}\NormalTok{(}\DataTypeTok{Theoretical_Mean =} \StringTok{"green"}\NormalTok{, }
                                                            \DataTypeTok{Sample_Mean =} \StringTok{"magenta"}\NormalTok{, }
                                                            \DataTypeTok{Normal_Distribution =} \StringTok{"red"}\NormalTok{)) }\OperatorTok{+}
\StringTok{      }\KeywordTok{ggtitle}\NormalTok{(}\StringTok{"Distribution of Exponential Means"}\NormalTok{) }\OperatorTok{+}
\StringTok{      }\KeywordTok{annotate}\NormalTok{(}\StringTok{"text"}\NormalTok{,}\DataTypeTok{label=}\KeywordTok{paste}\NormalTok{(}\StringTok{"Sample Mean ="}\NormalTok{,}\KeywordTok{round}\NormalTok{(}\KeywordTok{mean}\NormalTok{(mySampleStats}\OperatorTok{$}\NormalTok{Means),}\DecValTok{3}\NormalTok{)),}
               \DataTypeTok{x=}\DecValTok{6}\NormalTok{,}\DataTypeTok{y=}\DecValTok{100}\NormalTok{,}\DataTypeTok{size=}\DecValTok{5}\NormalTok{,}\DataTypeTok{color=}\StringTok{"magenta"}\NormalTok{)}\OperatorTok{+}
\StringTok{      }\KeywordTok{annotate}\NormalTok{(}\StringTok{"text"}\NormalTok{,}\DataTypeTok{label=}\KeywordTok{paste}\NormalTok{(}\StringTok{"Theoretical Mean ="}\NormalTok{,}\DecValTok{1}\OperatorTok{/}\NormalTok{.}\DecValTok{2}\NormalTok{),}
               \DataTypeTok{x=}\DecValTok{6}\NormalTok{,}\DataTypeTok{y=}\DecValTok{95}\NormalTok{,}\DataTypeTok{size=}\DecValTok{5}\NormalTok{,}\DataTypeTok{color=}\StringTok{"green"}\NormalTok{)}\OperatorTok{+}
\StringTok{      }\KeywordTok{annotate}\NormalTok{(}\StringTok{"text"}\NormalTok{,}\DataTypeTok{label=}\KeywordTok{paste}\NormalTok{(}\StringTok{"Normal Distribution : N(mu="}\NormalTok{,}\KeywordTok{round}\NormalTok{(}\KeywordTok{mean}\NormalTok{(mySampleStats}\OperatorTok{$}\NormalTok{Means),}\DecValTok{3}\NormalTok{),}
                                  \StringTok{", sd ="}\NormalTok{,}\KeywordTok{round}\NormalTok{(}\KeywordTok{sqrt}\NormalTok{(}\KeywordTok{var}\NormalTok{(mySampleStats}\OperatorTok{$}\NormalTok{Means)),}\DecValTok{3}\NormalTok{),}\StringTok{")"}\NormalTok{),}
               \DataTypeTok{x=}\FloatTok{6.65}\NormalTok{,}\DataTypeTok{y=}\DecValTok{90}\NormalTok{,}\DataTypeTok{size=}\DecValTok{5}\NormalTok{,}\DataTypeTok{color=}\StringTok{"red"}\NormalTok{)}
\end{Highlighting}
\end{Shaded}

\includegraphics{Statistical_Inference_Assignment_files/figure-latex/MeanCompare-1.pdf}

It can be seen from the figure above that the Central Limit Theorem
applies well to our exponential sample means. The Theoretical Mean is
quite close to the Sample Mean and the shape of the histogram of the
means is very
\href{https://en.wikipedia.org/wiki/Normal_distribution}{Gaussian}. This
distribution is approximately normal.

\subsubsection{Sample Variance versus Theoretical
Variance}\label{sample-variance-versus-theoretical-variance}

\begin{Shaded}
\begin{Highlighting}[]
\KeywordTok{ggplot}\NormalTok{(mySampleStats, }\KeywordTok{aes}\NormalTok{(}\DataTypeTok{x =}\NormalTok{ Variances, }\DataTypeTok{mean =} \KeywordTok{mean}\NormalTok{(mySampleStats}\OperatorTok{$}\NormalTok{Variances), }
                          \DataTypeTok{sd =} \KeywordTok{sqrt}\NormalTok{(}\KeywordTok{var}\NormalTok{(mySampleStats}\OperatorTok{$}\NormalTok{Variances)), }
                          \DataTypeTok{binwidth =} \DecValTok{2}\NormalTok{, }\DataTypeTok{n =} \KeywordTok{nrow}\NormalTok{(mySampleStats))) }\OperatorTok{+}
\StringTok{      }\KeywordTok{theme_dark}\NormalTok{() }\OperatorTok{+}
\StringTok{      }\KeywordTok{geom_histogram}\NormalTok{(}\DataTypeTok{binwidth =} \DecValTok{2}\NormalTok{, }\DataTypeTok{colour =} \StringTok{"white"}\NormalTok{, }\DataTypeTok{fill =} \StringTok{"blue"}\NormalTok{, }\DataTypeTok{size =} \FloatTok{0.1}\NormalTok{) }\OperatorTok{+}
\StringTok{      }\KeywordTok{stat_function}\NormalTok{(}\DataTypeTok{fun =} \ControlFlowTok{function}\NormalTok{(x) }\KeywordTok{dnorm}\NormalTok{(x, }\DataTypeTok{mean =} \KeywordTok{mean}\NormalTok{(mySampleStats}\OperatorTok{$}\NormalTok{Variances), }
                                            \DataTypeTok{sd =} \KeywordTok{sqrt}\NormalTok{(}\KeywordTok{var}\NormalTok{(mySampleStats}\OperatorTok{$}\NormalTok{Variances))) }\OperatorTok{*}\StringTok{ }\KeywordTok{nrow}\NormalTok{(mySampleStats) }\OperatorTok{*}\StringTok{ }\DecValTok{2}\NormalTok{,}
                    \KeywordTok{aes}\NormalTok{(}\DataTypeTok{color =} \StringTok{"Normal_Distribution"}\NormalTok{), }\DataTypeTok{size =} \DecValTok{1}\NormalTok{) }\OperatorTok{+}
\StringTok{      }\KeywordTok{geom_vline}\NormalTok{(}\KeywordTok{aes}\NormalTok{(}\DataTypeTok{xintercept =}\NormalTok{ (}\DecValTok{1}\OperatorTok{/}\NormalTok{.}\DecValTok{2}\NormalTok{)}\OperatorTok{^}\DecValTok{2}\NormalTok{, }\DataTypeTok{color =} \StringTok{"Theoretical_Variance"}\NormalTok{), }\DataTypeTok{linetype=}\StringTok{"dashed"}\NormalTok{, }\DataTypeTok{size=}\FloatTok{1.5}\NormalTok{) }\OperatorTok{+}
\StringTok{      }\KeywordTok{geom_vline}\NormalTok{(}\KeywordTok{aes}\NormalTok{(}\DataTypeTok{xintercept =} \KeywordTok{mean}\NormalTok{(mySampleStats}\OperatorTok{$}\NormalTok{Variances), }\DataTypeTok{color =} \StringTok{"Sample_Variance"}\NormalTok{), }\DataTypeTok{size=}\DecValTok{1}\NormalTok{) }\OperatorTok{+}
\StringTok{      }\KeywordTok{scale_color_manual}\NormalTok{(}\DataTypeTok{name =} \StringTok{"Chart Element"}\NormalTok{, }\DataTypeTok{values =} \KeywordTok{c}\NormalTok{(}\DataTypeTok{Theoretical_Variance =} \StringTok{"green"}\NormalTok{, }
                                                            \DataTypeTok{Sample_Variance =} \StringTok{"magenta"}\NormalTok{, }
                                                            \DataTypeTok{Normal_Distribution =} \StringTok{"red"}\NormalTok{)) }\OperatorTok{+}
\StringTok{      }\KeywordTok{ggtitle}\NormalTok{(}\StringTok{"Distribution of Exponential Variances"}\NormalTok{) }\OperatorTok{+}
\StringTok{      }\KeywordTok{annotate}\NormalTok{(}\StringTok{"text"}\NormalTok{,}\DataTypeTok{label=}\KeywordTok{paste}\NormalTok{(}\StringTok{"Sample Variance ="}\NormalTok{,}\KeywordTok{round}\NormalTok{(}\KeywordTok{mean}\NormalTok{(mySampleStats}\OperatorTok{$}\NormalTok{Variances),}\DecValTok{3}\NormalTok{)),}
               \DataTypeTok{x=}\DecValTok{40}\NormalTok{,}\DataTypeTok{y=}\DecValTok{90}\NormalTok{,}\DataTypeTok{size=}\DecValTok{5}\NormalTok{,}\DataTypeTok{color=}\StringTok{"magenta"}\NormalTok{)}\OperatorTok{+}
\StringTok{      }\KeywordTok{annotate}\NormalTok{(}\StringTok{"text"}\NormalTok{,}\DataTypeTok{label=}\KeywordTok{paste}\NormalTok{(}\StringTok{"Theoretical Variance ="}\NormalTok{,(}\DecValTok{1}\OperatorTok{/}\NormalTok{.}\DecValTok{2}\NormalTok{)}\OperatorTok{^}\DecValTok{2}\NormalTok{),}
               \DataTypeTok{x=}\DecValTok{40}\NormalTok{,}\DataTypeTok{y=}\DecValTok{85}\NormalTok{,}\DataTypeTok{size=}\DecValTok{5}\NormalTok{,}\DataTypeTok{color=}\StringTok{"green"}\NormalTok{) }\OperatorTok{+}
\StringTok{      }\KeywordTok{annotate}\NormalTok{(}\StringTok{"text"}\NormalTok{,}\DataTypeTok{label=}\KeywordTok{paste}\NormalTok{(}\StringTok{"Normal Distribution : N(mu="}\NormalTok{,}\KeywordTok{round}\NormalTok{(}\KeywordTok{mean}\NormalTok{(mySampleStats}\OperatorTok{$}\NormalTok{Variances),}\DecValTok{3}\NormalTok{),}
                                  \StringTok{", sd ="}\NormalTok{,}\KeywordTok{round}\NormalTok{(}\KeywordTok{sqrt}\NormalTok{(}\KeywordTok{var}\NormalTok{(mySampleStats}\OperatorTok{$}\NormalTok{Variances)),}\DecValTok{3}\NormalTok{),}\StringTok{")"}\NormalTok{),}
               \DataTypeTok{x=}\DecValTok{50}\NormalTok{,}\DataTypeTok{y=}\DecValTok{80}\NormalTok{,}\DataTypeTok{size=}\DecValTok{5}\NormalTok{,}\DataTypeTok{color=}\StringTok{"red"}\NormalTok{)}
\end{Highlighting}
\end{Shaded}

\includegraphics{Statistical_Inference_Assignment_files/figure-latex/VarianceCompare-1.pdf}

Comparing the figure above to the equivalent figure for the means shows
that the Central Limit Theorem does not apply as well to the variance of
our exponential samples. The Theoretical Variance is still close to the
Sample Variance, but the shape of the histogram of the variances is not
nearly as Gaussian shaped as the histogram of the means was. Perhaps
this is due to the unit scale of the variance relative to the mean. We
will attempt to reduce some of the skew shown in the above figure by
instead plotting the standard deviations of each of the samples.

\subsubsection{Sample StandardDeviation versus Theoretical
StandardDeviation}\label{sample-standarddeviation-versus-theoretical-standarddeviation}

\begin{Shaded}
\begin{Highlighting}[]
\KeywordTok{ggplot}\NormalTok{(mySampleStats, }\KeywordTok{aes}\NormalTok{(}\DataTypeTok{x =} \KeywordTok{sqrt}\NormalTok{(Variances), }\DataTypeTok{mean =} \KeywordTok{mean}\NormalTok{(}\KeywordTok{sqrt}\NormalTok{(mySampleStats}\OperatorTok{$}\NormalTok{Variances)), }
                          \DataTypeTok{sd =} \KeywordTok{sqrt}\NormalTok{(}\KeywordTok{var}\NormalTok{(}\KeywordTok{sqrt}\NormalTok{(mySampleStats}\OperatorTok{$}\NormalTok{Variances))),}
                          \DataTypeTok{binwidth =}\NormalTok{ .}\DecValTok{2}\NormalTok{, }\DataTypeTok{n =} \KeywordTok{nrow}\NormalTok{(mySampleStats))) }\OperatorTok{+}
\StringTok{      }\KeywordTok{theme_dark}\NormalTok{() }\OperatorTok{+}
\StringTok{      }\KeywordTok{geom_histogram}\NormalTok{(}\DataTypeTok{binwidth =}\NormalTok{ .}\DecValTok{2}\NormalTok{, }\DataTypeTok{colour =} \StringTok{"white"}\NormalTok{, }\DataTypeTok{fill =} \StringTok{"blue"}\NormalTok{, }\DataTypeTok{size =} \FloatTok{0.1}\NormalTok{) }\OperatorTok{+}
\StringTok{      }\KeywordTok{stat_function}\NormalTok{(}\DataTypeTok{fun =} \ControlFlowTok{function}\NormalTok{(x) }\KeywordTok{dnorm}\NormalTok{(x, }\DataTypeTok{mean =} \KeywordTok{mean}\NormalTok{(}\KeywordTok{sqrt}\NormalTok{(mySampleStats}\OperatorTok{$}\NormalTok{Variances)), }
                                            \DataTypeTok{sd =} \KeywordTok{sqrt}\NormalTok{(}\KeywordTok{var}\NormalTok{(}\KeywordTok{sqrt}\NormalTok{(mySampleStats}\OperatorTok{$}\NormalTok{Variances)))) }\OperatorTok{*}\StringTok{ }
\StringTok{                                                 }\KeywordTok{nrow}\NormalTok{(mySampleStats) }\OperatorTok{*}\StringTok{ }\NormalTok{.}\DecValTok{2}\NormalTok{,}
                    \KeywordTok{aes}\NormalTok{(}\DataTypeTok{color =} \StringTok{"Normal_Distribution"}\NormalTok{), }\DataTypeTok{size =} \DecValTok{1}\NormalTok{) }\OperatorTok{+}
\StringTok{      }\KeywordTok{geom_vline}\NormalTok{(}\KeywordTok{aes}\NormalTok{(}\DataTypeTok{xintercept =}\NormalTok{ (}\DecValTok{1}\OperatorTok{/}\NormalTok{.}\DecValTok{2}\NormalTok{), }\DataTypeTok{color =} \StringTok{"Theoretical_Standard_Deviation"}\NormalTok{), }\DataTypeTok{linetype=}\StringTok{"dashed"}\NormalTok{, }
                 \DataTypeTok{size=}\FloatTok{1.5}\NormalTok{) }\OperatorTok{+}
\StringTok{      }\KeywordTok{geom_vline}\NormalTok{(}\KeywordTok{aes}\NormalTok{(}\DataTypeTok{xintercept =} \KeywordTok{mean}\NormalTok{(}\KeywordTok{sqrt}\NormalTok{(mySampleStats}\OperatorTok{$}\NormalTok{Variances)), }\DataTypeTok{color =} \StringTok{"Sample_Standard_Deviation"}\NormalTok{), }
                 \DataTypeTok{size=}\DecValTok{1}\NormalTok{) }\OperatorTok{+}
\StringTok{      }\KeywordTok{scale_color_manual}\NormalTok{(}\DataTypeTok{name =} \StringTok{"Chart Element"}\NormalTok{, }\DataTypeTok{values =} \KeywordTok{c}\NormalTok{(}\DataTypeTok{Theoretical_Standard_Deviation =} \StringTok{"green"}\NormalTok{, }
                                                            \DataTypeTok{Sample_Standard_Deviation =} \StringTok{"magenta"}\NormalTok{, }
                                                            \DataTypeTok{Normal_Distribution =} \StringTok{"red"}\NormalTok{)) }\OperatorTok{+}
\StringTok{      }\KeywordTok{ggtitle}\NormalTok{(}\StringTok{"Distribution of Exponential Standard Deviations"}\NormalTok{)}\OperatorTok{+}
\StringTok{      }\KeywordTok{annotate}\NormalTok{(}\StringTok{"text"}\NormalTok{,}\DataTypeTok{label=}\KeywordTok{paste}\NormalTok{(}\StringTok{"Sample Std Dev ="}\NormalTok{,}\KeywordTok{round}\NormalTok{(}\KeywordTok{mean}\NormalTok{(}\KeywordTok{sqrt}\NormalTok{(mySampleStats}\OperatorTok{$}\NormalTok{Variances)),}\DecValTok{3}\NormalTok{)),}
               \DataTypeTok{x=}\FloatTok{6.05}\NormalTok{,}\DataTypeTok{y=}\DecValTok{100}\NormalTok{,}\DataTypeTok{size=}\DecValTok{5}\NormalTok{,}\DataTypeTok{color=}\StringTok{"magenta"}\NormalTok{) }\OperatorTok{+}
\StringTok{      }\KeywordTok{annotate}\NormalTok{(}\StringTok{"text"}\NormalTok{,}\DataTypeTok{label=}\KeywordTok{paste}\NormalTok{(}\StringTok{"Theoretical Std Dev ="}\NormalTok{,}\DecValTok{1}\OperatorTok{/}\NormalTok{.}\DecValTok{2}\NormalTok{),}
               \DataTypeTok{x=}\DecValTok{6}\NormalTok{,}\DataTypeTok{y=}\DecValTok{95}\NormalTok{,}\DataTypeTok{size=}\DecValTok{5}\NormalTok{,}\DataTypeTok{color=}\StringTok{"green"}\NormalTok{)}\OperatorTok{+}
\StringTok{      }\KeywordTok{annotate}\NormalTok{(}\StringTok{"text"}\NormalTok{,}\DataTypeTok{label=}\KeywordTok{paste}\NormalTok{(}\StringTok{"Normal Distribution : N(mu="}\NormalTok{,}\KeywordTok{round}\NormalTok{(}\KeywordTok{mean}\NormalTok{(}\KeywordTok{sqrt}\NormalTok{(mySampleStats}\OperatorTok{$}\NormalTok{Variances)),}\DecValTok{3}\NormalTok{),}
                                  \StringTok{", sd ="}\NormalTok{,}\KeywordTok{round}\NormalTok{(}\KeywordTok{sqrt}\NormalTok{(}\KeywordTok{var}\NormalTok{(}\KeywordTok{sqrt}\NormalTok{(mySampleStats}\OperatorTok{$}\NormalTok{Variances))),}\DecValTok{3}\NormalTok{),}\StringTok{")"}\NormalTok{),}
               \DataTypeTok{x=}\DecValTok{7}\NormalTok{,}\DataTypeTok{y=}\DecValTok{90}\NormalTok{,}\DataTypeTok{size=}\DecValTok{5}\NormalTok{,}\DataTypeTok{color=}\StringTok{"red"}\NormalTok{)}
\end{Highlighting}
\end{Shaded}

\includegraphics{Statistical_Inference_Assignment_files/figure-latex/StdDevCompare-1.pdf}

The Standard Deviation distribution appears to be quite Gaussian and is
approximately normal. This means that the Central Limit Theorem can be
applied to summary statistics other than just the mean.

\subsection{Part 2: ToothGrowth Inferential Data
Analysis}\label{part-2-toothgrowth-inferential-data-analysis}

\subsubsection{ToothGrowth Data
Summaries}\label{toothgrowth-data-summaries}

\begin{Shaded}
\begin{Highlighting}[]
\CommentTok{#~~~~~~~~~~~~~~~~~~~~~~~~RESAMPLING~~~~~~~~~~~~~~~~~~~~~~~~~~~~~~~~~~~~~~~~~~~~~~~~}

\KeywordTok{data}\NormalTok{(}\StringTok{"ToothGrowth"}\NormalTok{)}

\CommentTok{#Creating subsets for hypothesis testing and resampling}
\NormalTok{d.}\DecValTok{5}\NormalTok{<-}\KeywordTok{filter}\NormalTok{(ToothGrowth,dose}\OperatorTok{==}\NormalTok{.}\DecValTok{5}\NormalTok{)}
\NormalTok{d1<-}\KeywordTok{filter}\NormalTok{(ToothGrowth,dose}\OperatorTok{==}\DecValTok{1}\NormalTok{)}
\NormalTok{d2<-}\KeywordTok{filter}\NormalTok{(ToothGrowth,dose}\OperatorTok{==}\DecValTok{2}\NormalTok{)}
\NormalTok{sOJ<-}\KeywordTok{filter}\NormalTok{(ToothGrowth,supp}\OperatorTok{==}\StringTok{"OJ"}\NormalTok{)}
\NormalTok{sVC<-}\KeywordTok{filter}\NormalTok{(ToothGrowth,supp}\OperatorTok{==}\StringTok{"VC"}\NormalTok{)}

\CommentTok{#Summary for full dataset}
\KeywordTok{summary}\NormalTok{(ToothGrowth)}
\end{Highlighting}
\end{Shaded}

\begin{verbatim}
##       len        supp         dose      
##  Min.   : 4.20   OJ:30   Min.   :0.500  
##  1st Qu.:13.07   VC:30   1st Qu.:0.500  
##  Median :19.25           Median :1.000  
##  Mean   :18.81           Mean   :1.167  
##  3rd Qu.:25.27           3rd Qu.:2.000  
##  Max.   :33.90           Max.   :2.000
\end{verbatim}

\begin{Shaded}
\begin{Highlighting}[]
\CommentTok{#Summary for supplement dose of .5 mg}
\KeywordTok{summary}\NormalTok{(d.}\DecValTok{5}\NormalTok{)}
\end{Highlighting}
\end{Shaded}

\begin{verbatim}
##       len         supp         dose    
##  Min.   : 4.200   OJ:10   Min.   :0.5  
##  1st Qu.: 7.225   VC:10   1st Qu.:0.5  
##  Median : 9.850           Median :0.5  
##  Mean   :10.605           Mean   :0.5  
##  3rd Qu.:12.250           3rd Qu.:0.5  
##  Max.   :21.500           Max.   :0.5
\end{verbatim}

\begin{Shaded}
\begin{Highlighting}[]
\CommentTok{#Summary for supplement dose of 1 mg}
\KeywordTok{summary}\NormalTok{(d1)}
\end{Highlighting}
\end{Shaded}

\begin{verbatim}
##       len        supp         dose  
##  Min.   :13.60   OJ:10   Min.   :1  
##  1st Qu.:16.25   VC:10   1st Qu.:1  
##  Median :19.25           Median :1  
##  Mean   :19.73           Mean   :1  
##  3rd Qu.:23.38           3rd Qu.:1  
##  Max.   :27.30           Max.   :1
\end{verbatim}

\begin{Shaded}
\begin{Highlighting}[]
\CommentTok{#Summary for supplement dose of 2 mg}
\KeywordTok{summary}\NormalTok{(d2)}
\end{Highlighting}
\end{Shaded}

\begin{verbatim}
##       len        supp         dose  
##  Min.   :18.50   OJ:10   Min.   :2  
##  1st Qu.:23.52   VC:10   1st Qu.:2  
##  Median :25.95           Median :2  
##  Mean   :26.10           Mean   :2  
##  3rd Qu.:27.82           3rd Qu.:2  
##  Max.   :33.90           Max.   :2
\end{verbatim}

\begin{Shaded}
\begin{Highlighting}[]
\CommentTok{#Summary for supplement OJ}
\KeywordTok{summary}\NormalTok{(sOJ)}
\end{Highlighting}
\end{Shaded}

\begin{verbatim}
##       len        supp         dose      
##  Min.   : 8.20   OJ:30   Min.   :0.500  
##  1st Qu.:15.53   VC: 0   1st Qu.:0.500  
##  Median :22.70           Median :1.000  
##  Mean   :20.66           Mean   :1.167  
##  3rd Qu.:25.73           3rd Qu.:2.000  
##  Max.   :30.90           Max.   :2.000
\end{verbatim}

\begin{Shaded}
\begin{Highlighting}[]
\CommentTok{#Summary for supplement VC}
\KeywordTok{summary}\NormalTok{(sVC)}
\end{Highlighting}
\end{Shaded}

\begin{verbatim}
##       len        supp         dose      
##  Min.   : 4.20   OJ: 0   Min.   :0.500  
##  1st Qu.:11.20   VC:30   1st Qu.:0.500  
##  Median :16.50           Median :1.000  
##  Mean   :16.96           Mean   :1.167  
##  3rd Qu.:23.10           3rd Qu.:2.000  
##  Max.   :33.90           Max.   :2.000
\end{verbatim}

\begin{Shaded}
\begin{Highlighting}[]
\NormalTok{p1<-}\KeywordTok{ggplot}\NormalTok{(ToothGrowth, }\KeywordTok{aes}\NormalTok{(}\DataTypeTok{x=}\NormalTok{ supp, }\DataTypeTok{y=}\NormalTok{len, }\DataTypeTok{color=}\NormalTok{dose), }\DataTypeTok{geom =} \StringTok{"dotplot"}\NormalTok{) }\OperatorTok{+}\StringTok{ }\KeywordTok{geom_point}\NormalTok{() }\OperatorTok{+}\StringTok{ }
\StringTok{      }\KeywordTok{ggtitle}\NormalTok{(}\StringTok{"Supplement vs. Tooth Growth"}\NormalTok{)}
\NormalTok{p2<-}\KeywordTok{ggplot}\NormalTok{(ToothGrowth, }\KeywordTok{aes}\NormalTok{(}\DataTypeTok{x=}\NormalTok{ dose, }\DataTypeTok{y=}\NormalTok{len, }\DataTypeTok{color=}\NormalTok{supp), }\DataTypeTok{geom =} \StringTok{"dotplot"}\NormalTok{) }\OperatorTok{+}\StringTok{ }\KeywordTok{geom_point}\NormalTok{() }\OperatorTok{+}\StringTok{ }
\StringTok{      }\KeywordTok{ggtitle}\NormalTok{(}\StringTok{"Supplement Dosage vs. Tooth Growth"}\NormalTok{)}
\KeywordTok{grid.arrange}\NormalTok{(p1, p2, }\DataTypeTok{ncol =} \DecValTok{2}\NormalTok{)}
\end{Highlighting}
\end{Shaded}

\includegraphics{Statistical_Inference_Assignment_files/figure-latex/ToothGrowth Summaries-1.pdf}

In the Supplement vs.~Tooth Growth plot above, we can see that the range
for the OJ supplement lies entirely within the range of the VC
supplement. This will likely cause problems when trying to differentiate
the impacts of each of the supplements on tooth growth.\\
In the Supplement Dosage vs.~Tooth Growth plot above, so can see a
noticeable shift in tooth gowth as the dosage increases, although there
is still a noticeable amount of overlap in the ranges.\\
Conducting a quick
\href{https://en.wikipedia.org/wiki/Resampling_(statistics)}{resampling}
exercise should help to clear up whether the distributions are in fact
distinct or not.

\subsubsection{ToothGrowth Resampling
Visualizations}\label{toothgrowth-resampling-visualizations}

\begin{Shaded}
\begin{Highlighting}[]
\NormalTok{reSample_Dose<-}\ControlFlowTok{function}\NormalTok{(}\DataTypeTok{n=}\DecValTok{1000}\NormalTok{,}\DataTypeTok{yVar=}\DecValTok{3}\NormalTok{)\{}
\NormalTok{      rows.}\DecValTok{5}\NormalTok{<-}\KeywordTok{sample}\NormalTok{(}\DecValTok{1}\OperatorTok{:}\KeywordTok{nrow}\NormalTok{(d.}\DecValTok{5}\NormalTok{),}\DataTypeTok{size=}\NormalTok{n,}\DataTypeTok{replace=}\OtherTok{TRUE}\NormalTok{)}
\NormalTok{      rows1<-}\KeywordTok{sample}\NormalTok{(}\DecValTok{1}\OperatorTok{:}\KeywordTok{nrow}\NormalTok{(d1),}\DataTypeTok{size=}\NormalTok{n,}\DataTypeTok{replace=}\OtherTok{TRUE}\NormalTok{)}
\NormalTok{      rows2<-}\KeywordTok{sample}\NormalTok{(}\DecValTok{1}\OperatorTok{:}\KeywordTok{nrow}\NormalTok{(d2),}\DataTypeTok{size=}\NormalTok{n,}\DataTypeTok{replace=}\OtherTok{TRUE}\NormalTok{)}
      
\NormalTok{      dSamp.}\DecValTok{5}\NormalTok{<-d.}\DecValTok{5}\NormalTok{[rows.}\DecValTok{5}\NormalTok{,}\KeywordTok{c}\NormalTok{(}\DecValTok{1}\NormalTok{,yVar)]}
\NormalTok{      dSamp1<-d1[rows1,}\KeywordTok{c}\NormalTok{(}\DecValTok{1}\NormalTok{,yVar)]}
\NormalTok{      dSamp2<-d2[rows2,}\KeywordTok{c}\NormalTok{(}\DecValTok{1}\NormalTok{,yVar)]}
      
\NormalTok{      d.5_avg<-}\KeywordTok{mean}\NormalTok{(dSamp.}\DecValTok{5}\OperatorTok{$}\NormalTok{len)}
\NormalTok{      d1_avg<-}\KeywordTok{mean}\NormalTok{(dSamp1}\OperatorTok{$}\NormalTok{len)}
\NormalTok{      d2_avg<-}\KeywordTok{mean}\NormalTok{(dSamp2}\OperatorTok{$}\NormalTok{len)}
      
      \KeywordTok{return}\NormalTok{(}\KeywordTok{c}\NormalTok{(d.5_avg,d1_avg,d2_avg))}
\NormalTok{\}}

\NormalTok{mySample<-}\KeywordTok{data.frame}\NormalTok{(}\KeywordTok{t}\NormalTok{(}\KeywordTok{replicate}\NormalTok{(}\DecValTok{10000}\NormalTok{,}\KeywordTok{reSample_Dose}\NormalTok{(}\DataTypeTok{n=}\DecValTok{20}\NormalTok{))))}
\KeywordTok{names}\NormalTok{(mySample) <-}\StringTok{ }\KeywordTok{c}\NormalTok{(}\StringTok{"dose.5"}\NormalTok{,}\StringTok{"dose1"}\NormalTok{,}\StringTok{"dose2"}\NormalTok{)}
\NormalTok{mySampleMelt<-}\KeywordTok{melt}\NormalTok{(mySample)}
\KeywordTok{names}\NormalTok{(mySampleMelt)<-}\KeywordTok{c}\NormalTok{(}\StringTok{"dose"}\NormalTok{,}\StringTok{"tooth.length"}\NormalTok{)}
\KeywordTok{ggplot}\NormalTok{(mySampleMelt,}\KeywordTok{aes}\NormalTok{(}\DataTypeTok{x=}\NormalTok{tooth.length, }\DataTypeTok{color =}\NormalTok{ dose)) }\OperatorTok{+}
\StringTok{      }\KeywordTok{geom_histogram}\NormalTok{(}\DataTypeTok{binwidth =}\NormalTok{ .}\DecValTok{01}\NormalTok{) }\OperatorTok{+}
\StringTok{      }\KeywordTok{ggtitle}\NormalTok{(}\StringTok{"Distributions of Toothgrowth Length at Each Dosage"}\NormalTok{)}
\end{Highlighting}
\end{Shaded}

\includegraphics{Statistical_Inference_Assignment_files/figure-latex/Dosage_Resampling-1.pdf}

As we can see from the 10,000 resamples, it is likely that there is a
different average tooth growth at each dose level. There is very little
overlap in these resampled averages, so will probably reject a
hypothesis that the average tooth growth is the same across all dose
levels. A resample size of 20 was chosen for this exercise because there
are 20 data points for each dose level in the original dataset.

\begin{Shaded}
\begin{Highlighting}[]
\NormalTok{reSample_Supp<-}\ControlFlowTok{function}\NormalTok{(}\DataTypeTok{n=}\DecValTok{1000}\NormalTok{,}\DataTypeTok{yVar=}\DecValTok{2}\NormalTok{)\{}
\NormalTok{      rowsOJ<-}\KeywordTok{sample}\NormalTok{(}\DecValTok{1}\OperatorTok{:}\KeywordTok{nrow}\NormalTok{(sOJ),}\DataTypeTok{size=}\NormalTok{n,}\DataTypeTok{replace=}\OtherTok{TRUE}\NormalTok{)}
\NormalTok{      rowsVC<-}\KeywordTok{sample}\NormalTok{(}\DecValTok{1}\OperatorTok{:}\KeywordTok{nrow}\NormalTok{(sVC),}\DataTypeTok{size=}\NormalTok{n,}\DataTypeTok{replace=}\OtherTok{TRUE}\NormalTok{)}
      
\NormalTok{      sSampOJ<-sOJ[rowsOJ,}\KeywordTok{c}\NormalTok{(}\DecValTok{1}\NormalTok{,yVar)]}
\NormalTok{      sSampVC<-sVC[rowsVC,}\KeywordTok{c}\NormalTok{(}\DecValTok{1}\NormalTok{,yVar)]}
      
\NormalTok{      OJ_avg<-}\KeywordTok{mean}\NormalTok{(sSampOJ}\OperatorTok{$}\NormalTok{len)}
\NormalTok{      VC_avg<-}\KeywordTok{mean}\NormalTok{(sSampVC}\OperatorTok{$}\NormalTok{len)}
      
      \KeywordTok{return}\NormalTok{(}\KeywordTok{c}\NormalTok{(OJ_avg,VC_avg))}
\NormalTok{\}}

\NormalTok{mySample2<-}\KeywordTok{data.frame}\NormalTok{(}\KeywordTok{t}\NormalTok{(}\KeywordTok{replicate}\NormalTok{(}\DecValTok{10000}\NormalTok{,}\KeywordTok{reSample_Supp}\NormalTok{(}\DataTypeTok{n=}\DecValTok{30}\NormalTok{))))}
\KeywordTok{names}\NormalTok{(mySample2) <-}\StringTok{ }\KeywordTok{c}\NormalTok{(}\StringTok{"suppOJ"}\NormalTok{,}\StringTok{"suppVC"}\NormalTok{)}
\NormalTok{mySampleMelt2<-}\KeywordTok{melt}\NormalTok{(mySample2)}
\KeywordTok{names}\NormalTok{(mySampleMelt2)<-}\KeywordTok{c}\NormalTok{(}\StringTok{"supp"}\NormalTok{,}\StringTok{"tooth.length"}\NormalTok{)}
\KeywordTok{ggplot}\NormalTok{(mySampleMelt2,}\KeywordTok{aes}\NormalTok{(}\DataTypeTok{x=}\NormalTok{tooth.length, }\DataTypeTok{color =}\NormalTok{ supp)) }\OperatorTok{+}
\StringTok{      }\KeywordTok{geom_histogram}\NormalTok{(}\DataTypeTok{binwidth =}\NormalTok{ .}\DecValTok{01}\NormalTok{) }\OperatorTok{+}
\StringTok{      }\KeywordTok{ggtitle}\NormalTok{(}\StringTok{"Distributions of Toothgrowth Length for Each Supplement"}\NormalTok{)}
\end{Highlighting}
\end{Shaded}

\includegraphics{Statistical_Inference_Assignment_files/figure-latex/Supplement_Resampling-1.pdf}

As we can see from the 10,000 resamples, it is unlikely that we will
reject a hypothesis that the average tooth growth is the same across
both supplement levels. There is significant overlap in the resampled
tooth growth distributions for these two supplements. A resample size of
30 was chosen for this exercise because there are 30 data points for
each supplement level in the original dataset.

\begin{Shaded}
\begin{Highlighting}[]
\NormalTok{reSample_Supp<-}\ControlFlowTok{function}\NormalTok{(}\DataTypeTok{n=}\DecValTok{1000}\NormalTok{,}\DataTypeTok{yVar=}\DecValTok{2}\NormalTok{)\{}
\NormalTok{      rowsOJ<-}\KeywordTok{sample}\NormalTok{(}\DecValTok{1}\OperatorTok{:}\KeywordTok{nrow}\NormalTok{(sOJ),}\DataTypeTok{size=}\NormalTok{n,}\DataTypeTok{replace=}\OtherTok{TRUE}\NormalTok{)}
\NormalTok{      rowsVC<-}\KeywordTok{sample}\NormalTok{(}\DecValTok{1}\OperatorTok{:}\KeywordTok{nrow}\NormalTok{(sVC),}\DataTypeTok{size=}\NormalTok{n,}\DataTypeTok{replace=}\OtherTok{TRUE}\NormalTok{)}
      
\NormalTok{      sSampOJ<-sOJ[rowsOJ,}\KeywordTok{c}\NormalTok{(}\DecValTok{1}\NormalTok{,yVar)]}
\NormalTok{      sSampVC<-sVC[rowsVC,}\KeywordTok{c}\NormalTok{(}\DecValTok{1}\NormalTok{,yVar)]}
      
\NormalTok{      OJ_avg<-}\KeywordTok{mean}\NormalTok{(sSampOJ}\OperatorTok{$}\NormalTok{len)}
\NormalTok{      VC_avg<-}\KeywordTok{mean}\NormalTok{(sSampVC}\OperatorTok{$}\NormalTok{len)}
      
      \KeywordTok{return}\NormalTok{(}\KeywordTok{c}\NormalTok{(OJ_avg,VC_avg))}
\NormalTok{\}}

\NormalTok{mySample2<-}\KeywordTok{data.frame}\NormalTok{(}\KeywordTok{t}\NormalTok{(}\KeywordTok{replicate}\NormalTok{(}\DecValTok{10000}\NormalTok{,}\KeywordTok{reSample_Supp}\NormalTok{(}\DataTypeTok{n=}\DecValTok{300}\NormalTok{))))}
\KeywordTok{names}\NormalTok{(mySample2) <-}\StringTok{ }\KeywordTok{c}\NormalTok{(}\StringTok{"suppOJ"}\NormalTok{,}\StringTok{"suppVC"}\NormalTok{)}
\NormalTok{mySampleMelt2<-}\KeywordTok{melt}\NormalTok{(mySample2)}
\KeywordTok{names}\NormalTok{(mySampleMelt2)<-}\KeywordTok{c}\NormalTok{(}\StringTok{"supp"}\NormalTok{,}\StringTok{"tooth.length"}\NormalTok{)}
\KeywordTok{ggplot}\NormalTok{(mySampleMelt2,}\KeywordTok{aes}\NormalTok{(}\DataTypeTok{x=}\NormalTok{tooth.length, }\DataTypeTok{color =}\NormalTok{ supp)) }\OperatorTok{+}
\StringTok{      }\KeywordTok{geom_histogram}\NormalTok{(}\DataTypeTok{binwidth =}\NormalTok{ .}\DecValTok{01}\NormalTok{) }\OperatorTok{+}
\StringTok{      }\KeywordTok{ggtitle}\NormalTok{(}\StringTok{"Distributions of Toothgrowth Length for Each Supplement"}\NormalTok{)}
\end{Highlighting}
\end{Shaded}

\includegraphics{Statistical_Inference_Assignment_files/figure-latex/Supplement_Resampling_Part2-1.pdf}

As evidenced from this figure, it is likely that we would reject the
null hypothesis of the two supplements having the same average tooth
growth length \textbf{if our sample size was 10 times larger} than in
the original dataset, assuming that the same distribution of tooth
growth would result.

\subsubsection{ToothGrowth T-Testing}\label{toothgrowth-t-testing}

\paragraph{Assumptions:}\label{assumptions}

\begin{enumerate}
\def\labelenumi{\arabic{enumi}.}
\tightlist
\item
  The ToothGrowth dataset contains all available information.
\item
  The experiment conducted was done in a scientific manner.

  \begin{itemize}
  \tightlist
  \item
    This includes the assumption that there was no overlap in test
    subjects being given varying doses or varying supplements.
  \end{itemize}
\end{enumerate}

\begin{Shaded}
\begin{Highlighting}[]
\CommentTok{#~~~~~~~~~~~~~~~~~~~~~~~~HYPOTHESIS }\AlertTok{TESTING}\CommentTok{~~~~~~~~~~~~~~~~~~~~~~~~~~~~~~~~~~~~~~~~~~~~~~~~}
\CommentTok{#Null Hypothesis is: The average tooth growth is identical across all doses and all supplements}

\CommentTok{#T-testing the alternative hypothesis: The average tooth growth changes as supplement dosage changes.}
\KeywordTok{t.test}\NormalTok{(len}\OperatorTok{~}\NormalTok{dose,}\DataTypeTok{paired=}\OtherTok{FALSE}\NormalTok{,}\DataTypeTok{data=}\KeywordTok{rbind}\NormalTok{(d1,d.}\DecValTok{5}\NormalTok{))}
\end{Highlighting}
\end{Shaded}

\begin{verbatim}
## 
##  Welch Two Sample t-test
## 
## data:  len by dose
## t = -6.4766, df = 37.986, p-value = 1.268e-07
## alternative hypothesis: true difference in means is not equal to 0
## 95 percent confidence interval:
##  -11.983781  -6.276219
## sample estimates:
## mean in group 0.5   mean in group 1 
##            10.605            19.735
\end{verbatim}

\begin{Shaded}
\begin{Highlighting}[]
\KeywordTok{t.test}\NormalTok{(len}\OperatorTok{~}\NormalTok{dose,}\DataTypeTok{paired=}\OtherTok{FALSE}\NormalTok{,}\DataTypeTok{data=}\KeywordTok{rbind}\NormalTok{(d2,d1))}
\end{Highlighting}
\end{Shaded}

\begin{verbatim}
## 
##  Welch Two Sample t-test
## 
## data:  len by dose
## t = -4.9005, df = 37.101, p-value = 1.906e-05
## alternative hypothesis: true difference in means is not equal to 0
## 95 percent confidence interval:
##  -8.996481 -3.733519
## sample estimates:
## mean in group 1 mean in group 2 
##          19.735          26.100
\end{verbatim}

\begin{Shaded}
\begin{Highlighting}[]
\KeywordTok{t.test}\NormalTok{(len}\OperatorTok{~}\NormalTok{dose,}\DataTypeTok{paired=}\OtherTok{FALSE}\NormalTok{,}\DataTypeTok{data=}\KeywordTok{rbind}\NormalTok{(d2,d.}\DecValTok{5}\NormalTok{))}
\end{Highlighting}
\end{Shaded}

\begin{verbatim}
## 
##  Welch Two Sample t-test
## 
## data:  len by dose
## t = -11.799, df = 36.883, p-value = 4.398e-14
## alternative hypothesis: true difference in means is not equal to 0
## 95 percent confidence interval:
##  -18.15617 -12.83383
## sample estimates:
## mean in group 0.5   mean in group 2 
##            10.605            26.100
\end{verbatim}

\paragraph{Dosage Conclusions}\label{dosage-conclusions}

We tested each of the dosage subsets against each other, and with each
set of doses we can be 95\% confident that the average tooth growth is
different between the dose amounts. Our P value is much smaller than 5\%
in each case. Thus, we reject the null hypothesis in favor of the
alternative hypothesis that average tooth growth varies by supplement
dose amount.

\begin{Shaded}
\begin{Highlighting}[]
\CommentTok{#T-testing the alternative hypothesis: The average tooth growth changes as supplement changes.}
\KeywordTok{t.test}\NormalTok{(len}\OperatorTok{~}\NormalTok{supp,}\DataTypeTok{paired=}\OtherTok{FALSE}\NormalTok{,}\DataTypeTok{data=}\NormalTok{ToothGrowth)}
\end{Highlighting}
\end{Shaded}

\begin{verbatim}
## 
##  Welch Two Sample t-test
## 
## data:  len by supp
## t = 1.9153, df = 55.309, p-value = 0.06063
## alternative hypothesis: true difference in means is not equal to 0
## 95 percent confidence interval:
##  -0.1710156  7.5710156
## sample estimates:
## mean in group OJ mean in group VC 
##         20.66333         16.96333
\end{verbatim}

\paragraph{Supplement Conclusions}\label{supplement-conclusions}

We tested the two supplement subsets against each other, and we cannot
be 95\% confident that the average tooth growth is different between the
supplements. This is because:\\
1. Our P-Value for this test is greater than our alpha 5\% (1 - 95\%).\\
2. Our 95\% confidence interval contains the possibility of a 0
difference in average tooth growth as a function of supplement type.\\
Thus, we fail to reject the null hypothesis in favor of the alternative
hypothesis that average tooth growth varies by supplement.


\end{document}
